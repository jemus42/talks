%%%%%%%%%%%%%%%%%%%%%%%%%%%%%%%%%%%%%%%%%%%%%%%%%%%%%
% Primary document settings
%%%%%%%%%%%%%%%%%%%%%%%%%%%%%%%%%%%%%%%%%%%%%%%%%%%%%
\documentclass{beamer}
\usepackage{ifxetex,ifluatex}
\ifnum 0\ifxetex 1\fi\ifluatex 1\fi=0 % if pdftex
  \usepackage[T1]{fontenc}
  \usepackage[utf8]{inputenc}
  \usepackage{textcomp} % provide euro and other symbols
\else % if luatex or xetex
  \usepackage{unicode-math}
  \defaultfontfeatures{Scale=MatchLowercase}
  \defaultfontfeatures[\rmfamily]{Ligatures=TeX,Scale=1}

\usepackage[
    backend=biber,
    natbib=true,
    style=apa,
    %bibstyle=authoryear, citestyle=apa,
    %style=authoryear,
    sorting=nyt,
    %sortlocale=de_DE,
    sortlocale=en_US,
    url=false,
    doi=true,
]{biblatex}

\addbibresource{references.bib}

%%%%%%%%%%%%%%%%%%%%%%%%%%%%%%%%%%%%%%%%%%%%%%%%%%%%%%%%%%%%%%
% Extra stuff for Rmarkdown to work (code blocks)
%%%%%%%%%%%%%%%%%%%%%%%%%%%%%%%%%%%%%%%%%%%%%%%%%%%%%%%%%%%%%%
\providecommand{\tightlist}{%
  \setlength{\itemsep}{0pt}\setlength{\parskip}{0pt}}
\usepackage{framed}
\definecolor{shadecolor}{RGB}{248,248,248}
\newenvironment{Shaded}{\begin{snugshade}}{\end{snugshade}}
\usepackage{fancyvrb}
\newcommand{\VerbBar}{|}
\newcommand{\VERB}{\Verb[commandchars=\\\{\}]}
\DefineVerbatimEnvironment{verbatim}{Verbatim}{xleftmargin=0em}
\DefineVerbatimEnvironment{Highlighting}{Verbatim}{xleftmargin=0em,commandchars=\\\{\}}

\newcommand{\KeywordTok}[1]{\textcolor[rgb]{0.13,0.29,0.53}{\textbf{{#1}}}}
\newcommand{\DataTypeTok}[1]{\textcolor[rgb]{0.13,0.29,0.53}{{#1}}}
\newcommand{\DecValTok}[1]{\textcolor[rgb]{0.00,0.00,0.81}{{#1}}}
\newcommand{\BaseNTok}[1]{\textcolor[rgb]{0.00,0.00,0.81}{{#1}}}
\newcommand{\FloatTok}[1]{\textcolor[rgb]{0.00,0.00,0.81}{{#1}}}
\newcommand{\ConstantTok}[1]{\textcolor[rgb]{0.00,0.00,0.00}{{#1}}}
\newcommand{\CharTok}[1]{\textcolor[rgb]{0.31,0.60,0.02}{{#1}}}
\newcommand{\SpecialCharTok}[1]{\textcolor[rgb]{0.00,0.00,0.00}{{#1}}}
\newcommand{\StringTok}[1]{\textcolor[rgb]{0.31,0.60,0.02}{{#1}}}
\newcommand{\VerbatimStringTok}[1]{\textcolor[rgb]{0.31,0.60,0.02}{{#1}}}
\newcommand{\SpecialStringTok}[1]{\textcolor[rgb]{0.31,0.60,0.02}{{#1}}}
\newcommand{\ImportTok}[1]{{#1}}
\newcommand{\CommentTok}[1]{\textcolor[rgb]{0.56,0.35,0.01}{\textit{{#1}}}}
\newcommand{\DocumentationTok}[1]{\textcolor[rgb]{0.56,0.35,0.01}{\textbf{\textit{{#1}}}}}
\newcommand{\AnnotationTok}[1]{\textcolor[rgb]{0.56,0.35,0.01}{\textbf{\textit{{#1}}}}}
\newcommand{\CommentVarTok}[1]{\textcolor[rgb]{0.56,0.35,0.01}{\textbf{\textit{{#1}}}}}
\newcommand{\OtherTok}[1]{\textcolor[rgb]{0.56,0.35,0.01}{{#1}}}
\newcommand{\FunctionTok}[1]{\textcolor[rgb]{0.00,0.00,0.00}{{#1}}}
\newcommand{\VariableTok}[1]{\textcolor[rgb]{0.00,0.00,0.00}{{#1}}}
\newcommand{\ControlFlowTok}[1]{\textcolor[rgb]{0.13,0.29,0.53}{\textbf{{#1}}}}
\newcommand{\OperatorTok}[1]{\textcolor[rgb]{0.81,0.36,0.00}{\textbf{{#1}}}}
\newcommand{\BuiltInTok}[1]{{#1}}
\newcommand{\ExtensionTok}[1]{{#1}}
\newcommand{\PreprocessorTok}[1]{\textcolor[rgb]{0.56,0.35,0.01}{\textit{{#1}}}}
\newcommand{\AttributeTok}[1]{\textcolor[rgb]{0.77,0.63,0.00}{{#1}}}
\newcommand{\RegionMarkerTok}[1]{{#1}}
\newcommand{\InformationTok}[1]{\textcolor[rgb]{0.56,0.35,0.01}{\textbf{\textit{{#1}}}}}
\newcommand{\WarningTok}[1]{\textcolor[rgb]{0.56,0.35,0.01}{\textbf{\textit{{#1}}}}}
\newcommand{\AlertTok}[1]{\textcolor[rgb]{0.94,0.16,0.16}{{#1}}}
\newcommand{\ErrorTok}[1]{\textcolor[rgb]{0.64,0.00,0.00}{\textbf{{#1}}}}
\newcommand{\NormalTok}[1]{{#1}}

%%%%%%%%%%%%%%%%%%%%%%%%%%
% kableExtra stuff (tables)
%%%%%%%%%%%%%%%%%%%%%%%%%%
\usepackage{booktabs}
\usepackage{longtable}
\usepackage{array}
\usepackage{multirow}
\usepackage{xcolor}
\usepackage{wrapfig}
\usepackage{float}
\usepackage{colortbl}
\usepackage{pdflscape}
\usepackage{tabu}
\usepackage{threeparttable}
\usepackage{threeparttablex}
\usepackage[normalem]{ulem}
\usepackage{makecell}

%%%%%%%%%%%%%%%%%%%%%%%%%%
% Main document
%%%%%%%%%%%%%%%%%%%%%%%%%%
%\usepackage{bm}

\usetheme[fira]{BIPS}
%\usetheme[german]{BIPS}  % for the German version
%\usetheme[fira]{BIPS} % English with the Fira font
%\usetheme[german,fira]{BIPS} % German with the Fira font

% Note that for the Fira font, you need to use
% XeLaTeX instead of pdfLaTeX. You can find this in
% most interfaces

\title{Simulating Competing Risk Survival Settings}
\subtitle{With High-Dimensional Data}
\author{Lukas Burk}
\date{2023-05-13}
\contactauthor{Lukas Burk}
\occasion{BioWiMium}
\email{\href{mailto:burk@leibniz-bips.de}{\nolinkurl{burk@leibniz-bips.de}}}

\begin{document}

\frame{\maketitle}

\begin{frame}{Introduction}
\protect\hypertarget{introduction}{}
Simulation setup is part of an ongoing project.

\begin{block}{Motivation}
\protect\hypertarget{motivation}{}
\begin{itemize}
\tightlist
\item
  Variable selection in high-dimensional settings with competing risks
\item
  Example setting: Clinical + gene expression data (\(p >> 1000\))
\end{itemize}
\end{block}

\begin{block}{Outcome}
\protect\hypertarget{outcome}{}
\begin{itemize}
\tightlist
\item
  2 competing events, censoring approx. equal prevalence(!)
\end{itemize}
\end{block}
\end{frame}

\begin{frame}{Generating Survival Times (I)}
\protect\hypertarget{generating-survival-times-i}{}
\includegraphics[width=0.95\linewidth]{img/binder2009}

\textcite{binder2009boostinghighdimensional}
\end{frame}

\begin{frame}{Generating Survival Times (II)}
\protect\hypertarget{generating-survival-times-ii}{}
\includegraphics[width=0.95\linewidth]{img/binder2007}

\textcite{binder2008adaptingprediction}
\end{frame}

\begin{frame}{Generating Survival Times (III)}
\protect\hypertarget{generating-survival-times-iii}{}
\includegraphics[width=0.95\linewidth]{img/bender2003}

\textcite{bender2005generatingsurvival}
\end{frame}

\begin{frame}{Covariates: Structure}
\protect\hypertarget{covariates-structure}{}
\begin{itemize}
\tightlist
\item
  \(N = 400\), \(p = 5000\), 16 informative (12 per event)
\item
  Organized in 4 blocks of correlated variables \& uncorrelated noise
\end{itemize}

\begin{block}{Blocks}
\protect\hypertarget{blocks}{}
\begin{itemize}
\tightlist
\item
  Block 1: \(\rho \approx 0.5\)
\item
  Block 2: \(\rho \approx 0.35\)
\item
  Block 3: \(\rho \approx 0.05\)
\item
  Block 4: \(\rho \approx 0.32\)
\item
  Rest: \(\rho \approx 0\)
\end{itemize}
\end{block}
\end{frame}

\begin{frame}{Covariates: Generation}
\protect\hypertarget{covariates-generation}{}
\begin{itemize}
\tightlist
\item
  \(j = 1, \ldots, p\) and \(i = 1, \ldots, N\)
\item
  \(\epsilon_{ij} \sim \mathcal{N}(0, 1)\) and
  \(U_{i\{1,2,3\}} \sim \mathcal{U}(0, 1)\)
\end{itemize}

\[
x_{ij} = \begin{cases}
-1 + \epsilon_{ij}                                   & \text{for}\ i \leq 0.5n \text{ and } j \leq 0.05p \\
1 + \epsilon_{ij}                                    & \text{for}\ i > 0.5n \text{ and } j \leq 0.05p \\
1.5 \cdot \mathbf{1}\{U_{i1} < 0.4\} + \epsilon_{ij} & \text{for}\ 0.05p < j \leq 0.1p \\
0.5 \cdot \mathbf{1}\{U_{i2} < 0.7\} + \epsilon_{ij} & \text{for}\ 0.1p < j \leq 0.2p \\
1.5 \cdot \mathbf{1}\{U_{i3} < 0.3\} + \epsilon_{ij} & \text{for}\ 0.2p < j \leq 0.3p \\
\epsilon_{ij}                                        & \text{for}\ j > 0.3p
\end{cases}
\]
\end{frame}

\begin{frame}{Effect Assignment}
\protect\hypertarget{effect-assignment}{}
\begin{itemize}
\tightlist
\item
  Cause-specific hazards for events \(k = 1,2\) and coefficients
  \(\pmb{\beta}^{(k)}\)
\item
  \(\beta^{(k)}_j = \pm 0.5\) for effect variables
\end{itemize}

\begin{block}{Blocks}
\protect\hypertarget{blocks-1}{}
\begin{itemize}
\tightlist
\item
  Block 1 (\textbf{``Mutual''}):

  \begin{itemize}
  \tightlist
  \item
    4 variables with positive effect in both of \(\pmb{\beta}^{(1,2)}\)
  \end{itemize}
\item
  Block 2 (\textbf{``Reversed''}):

  \begin{itemize}
  \tightlist
  \item
    4 variables with positive effect in \(\pmb{\beta}^{(1)}\) and
    negative in \(\pmb{\beta}^{(2)}\)
  \end{itemize}
\item
  Block 3 (\textbf{``Disjoint''}):

  \begin{itemize}
  \tightlist
  \item
    3.1: 4 variables with negative effect in \(\pmb{\beta}^{(1)}\) only
  \item
    3.2: 4 (other) variables with positive effect in
    \(\pmb{\beta}^{(2)}\) only
  \end{itemize}
\end{itemize}
\end{block}
\end{frame}

\begin{frame}{Event and Censoring Times (I)}
\protect\hypertarget{event-and-censoring-times-i}{}
\begin{itemize}
\tightlist
\item
  Cox-exponential model for events \(k = 1,2\):
  \[\lambda^{(k)}(t) = \lambda_k \exp(\mathbf{x}^T \pmb{\beta}^{(k)})\]
\item
  \(\lambda_{1,2}\): Constant baseline hazards for event times
  \(T_i^{(k)}\)
\item
  \(\lambda_C\): Analogue for censoring times \(C_i\)
\item
  Default setting: \(\lambda_1 = \lambda_2 = \lambda_C = 0.1\)
\end{itemize}

\begin{align*}
T_i^{(k)} &= \frac{-\log(U_i)}{\lambda_k \exp(\mathbf{x}^{T}_i\pmb{\beta}^{(k)})} &
C_i       = \frac{-\log(U_i)}{\lambda_C}
\end{align*}

\[U_i \sim \mathcal{U}(0, 1) \quad \Longrightarrow \quad  -\log(U_i) \sim \mathrm{Exponential}(\lambda = 1)\]
\end{frame}

\begin{frame}{Event and Censoring Times (II)}
\protect\hypertarget{event-and-censoring-times-ii}{}
Assign observed event times \(t_i\) and censoring \(\delta_i\) indicator
accordingly

\begin{align*}
t_i &= \min(T_i^{(1)}, T_i^{(2)}, C_i) & 
\delta_i &= \begin{cases}
  0 & \text{if}\ t_i = C_i \\
  1 & \text{if}\ t_i = T_i^{(1)} \\
  2 & \text{if}\ t_i = T_i^{(2)} \\
  \end{cases}
\end{align*}
\end{frame}

\begin{frame}{Resulting Outcome}
\protect\hypertarget{resulting-outcome}{}
\begin{itemize}
\tightlist
\item
  \(\pmb{\beta}^{(k)}\) sparse with 12 non-zero entries,
  \(\sum_{i = 1}^p \beta^{(k)}_j = 2\)
\end{itemize}

\begin{itemize}
\tightlist
\item
  \(\exp(\mathbf{x}^{T}_i\pmb{\beta}^{(k)})\) in range of
  \([10^{-4}, 10^{3}]\)
\end{itemize}

\begin{block}{Event prevalences}
\protect\hypertarget{event-prevalences}{}
Mean event counts and prevalence after 100 replicates with \(N = 400\):

\begin{table}
\centering
\begin{tabular}{l|l|l}
\hline
$\delta$ & $n$ (min - max) & \% (min - max)  \\
\hline
0 & 118.0 (97 - 136) & 29.5\% (24.2\% - 34.0\%)\\
\hline
1 & 165.1 (142 - 190) & 41.3\% (35.5\% - 47.5\%)\\
\hline
2 & 116.9 (93 - 141) & 29.2\% (23.2\% - 35.2\%)\\
\hline
\end{tabular}
\end{table}
\end{block}
\end{frame}

\begin{frame}{Thanks for listening!}
\protect\hypertarget{thanks-for-listening}{}
Next up: Backup slides with code
\end{frame}

\begin{frame}[fragile]{Predictors: Code}
\protect\hypertarget{predictors-code}{}
\begin{Shaded}
\begin{Highlighting}[]
\NormalTok{  X }\OtherTok{\textless{}{-}} \FunctionTok{matrix}\NormalTok{(}\FunctionTok{rnorm}\NormalTok{(n }\SpecialCharTok{*}\NormalTok{ p), }\AttributeTok{nrow =}\NormalTok{ n, }\AttributeTok{ncol =}\NormalTok{ p)}
\NormalTok{  ui1 }\OtherTok{\textless{}{-}} \FunctionTok{runif}\NormalTok{(n); ui2 }\OtherTok{\textless{}{-}} \FunctionTok{runif}\NormalTok{(n); ui3 }\OtherTok{\textless{}{-}} \FunctionTok{runif}\NormalTok{(n)}
\NormalTok{  j\_seq }\OtherTok{\textless{}{-}} \FunctionTok{seq\_len}\NormalTok{(p)}
\NormalTok{  block1 }\OtherTok{\textless{}{-}} \FunctionTok{which}\NormalTok{(j\_seq }\SpecialCharTok{\textless{}=} \FloatTok{0.05} \SpecialCharTok{*}\NormalTok{ p)}
\NormalTok{  X[}\FunctionTok{seq\_len}\NormalTok{(n}\SpecialCharTok{/}\DecValTok{2}\NormalTok{), block1] }\OtherTok{\textless{}{-}} \SpecialCharTok{{-}}\DecValTok{1} \SpecialCharTok{+}\NormalTok{ X[}\FunctionTok{seq\_len}\NormalTok{(n}\SpecialCharTok{/}\DecValTok{2}\NormalTok{), block1]}
\NormalTok{  X[}\SpecialCharTok{{-}}\FunctionTok{seq\_len}\NormalTok{(n}\SpecialCharTok{/}\DecValTok{2}\NormalTok{), block1] }\OtherTok{\textless{}{-}} \DecValTok{1} \SpecialCharTok{+}\NormalTok{ X[}\SpecialCharTok{{-}}\FunctionTok{seq\_len}\NormalTok{(n}\SpecialCharTok{/}\DecValTok{2}\NormalTok{), block1]}
\NormalTok{  block2 }\OtherTok{\textless{}{-}} \FunctionTok{which}\NormalTok{((j\_seq }\SpecialCharTok{\textgreater{}}\NormalTok{ (}\FloatTok{0.05} \SpecialCharTok{*}\NormalTok{ p)) }\SpecialCharTok{\&}\NormalTok{ (j\_seq }\SpecialCharTok{\textless{}=}\NormalTok{ (}\FloatTok{0.1} \SpecialCharTok{*}\NormalTok{ p)))}
\NormalTok{  X[, block2] }\OtherTok{\textless{}{-}} \FloatTok{1.5} \SpecialCharTok{*}\NormalTok{ (ui1 }\SpecialCharTok{\textless{}} \FloatTok{0.4}\NormalTok{) }\SpecialCharTok{+}\NormalTok{ X[, block2]}
\NormalTok{  block3 }\OtherTok{\textless{}{-}} \FunctionTok{which}\NormalTok{((}\FloatTok{0.1} \SpecialCharTok{*}\NormalTok{ p }\SpecialCharTok{\textless{}}\NormalTok{ j\_seq) }\SpecialCharTok{\&}\NormalTok{ (j\_seq }\SpecialCharTok{\textless{}=} \FloatTok{0.2} \SpecialCharTok{*}\NormalTok{ p))}
\NormalTok{  X[, block3] }\OtherTok{\textless{}{-}} \FloatTok{0.5} \SpecialCharTok{*}\NormalTok{ (ui2 }\SpecialCharTok{\textless{}} \FloatTok{0.7}\NormalTok{) }\SpecialCharTok{+}\NormalTok{ X[, block3]}
\NormalTok{  block4 }\OtherTok{\textless{}{-}} \FunctionTok{which}\NormalTok{((}\FloatTok{0.2} \SpecialCharTok{*}\NormalTok{ p }\SpecialCharTok{\textless{}}\NormalTok{ j\_seq) }\SpecialCharTok{\&}\NormalTok{ (j\_seq }\SpecialCharTok{\textless{}=} \FloatTok{0.3} \SpecialCharTok{*}\NormalTok{ p))}
\NormalTok{  X[, block4] }\OtherTok{\textless{}{-}} \FloatTok{1.5} \SpecialCharTok{*}\NormalTok{ (ui3 }\SpecialCharTok{\textless{}} \FloatTok{0.3}\NormalTok{) }\SpecialCharTok{+}\NormalTok{ X[, block4]}
\end{Highlighting}
\end{Shaded}
\end{frame}

\begin{frame}[fragile]{Efect Assignment: Code}
\protect\hypertarget{efect-assignment-code}{}
\begin{Shaded}
\begin{Highlighting}[]
\NormalTok{  ce }\OtherTok{\textless{}{-}} \FloatTok{0.5}
  \CommentTok{\# first block}
\NormalTok{  j\_block1 }\OtherTok{\textless{}{-}} \FunctionTok{which}\NormalTok{(j\_seq }\SpecialCharTok{\textless{}=} \FloatTok{0.05} \SpecialCharTok{*}\NormalTok{ p)}
\NormalTok{  beta1[j\_block1[}\DecValTok{1}\SpecialCharTok{:}\DecValTok{4}\NormalTok{]] }\OtherTok{\textless{}{-}}\NormalTok{ ce}
\NormalTok{  beta2[j\_block1[}\DecValTok{1}\SpecialCharTok{:}\DecValTok{4}\NormalTok{]] }\OtherTok{\textless{}{-}}\NormalTok{ ce}
  \CommentTok{\# second block}
\NormalTok{  j\_block2 }\OtherTok{\textless{}{-}} \FunctionTok{which}\NormalTok{((j\_seq }\SpecialCharTok{\textgreater{}}\NormalTok{ (}\FloatTok{0.05} \SpecialCharTok{*}\NormalTok{ p)) }\SpecialCharTok{\&}\NormalTok{ (j\_seq }\SpecialCharTok{\textless{}=}\NormalTok{ (}\FloatTok{0.1} \SpecialCharTok{*}\NormalTok{ p)))}
\NormalTok{  beta1[j\_block2[}\DecValTok{1}\SpecialCharTok{:}\DecValTok{4}\NormalTok{]] }\OtherTok{\textless{}{-}}\NormalTok{ ce}
\NormalTok{  beta2[j\_block2[}\DecValTok{1}\SpecialCharTok{:}\DecValTok{4}\NormalTok{]] }\OtherTok{\textless{}{-}} \SpecialCharTok{{-}}\NormalTok{ce}
  \CommentTok{\# third block}
\NormalTok{  j\_block3 }\OtherTok{\textless{}{-}} \FunctionTok{which}\NormalTok{((}\FloatTok{0.1} \SpecialCharTok{*}\NormalTok{ p }\SpecialCharTok{\textless{}}\NormalTok{ j\_seq) }\SpecialCharTok{\&}\NormalTok{ (j\_seq }\SpecialCharTok{\textless{}=} \FloatTok{0.2} \SpecialCharTok{*}\NormalTok{ p))}
\NormalTok{  beta1[j\_block3[}\DecValTok{1}\SpecialCharTok{:}\DecValTok{4}\NormalTok{]] }\OtherTok{\textless{}{-}} \SpecialCharTok{{-}}\NormalTok{ce}
\NormalTok{  beta2[j\_block3[}\DecValTok{5}\SpecialCharTok{:}\DecValTok{8}\NormalTok{]] }\OtherTok{\textless{}{-}}\NormalTok{ ce }\CommentTok{\# offset by 4}
\end{Highlighting}
\end{Shaded}
\end{frame}

\begin{frame}[fragile]{Event Times: Code}
\protect\hypertarget{event-times-code}{}
\begin{Shaded}
\begin{Highlighting}[]
\NormalTok{  lp1 }\OtherTok{\textless{}{-}}\NormalTok{ X }\SpecialCharTok{\%*\%}\NormalTok{ beta1}
\NormalTok{  lp2 }\OtherTok{\textless{}{-}}\NormalTok{ X }\SpecialCharTok{\%*\%}\NormalTok{ beta2}

\NormalTok{  Ti1 }\OtherTok{\textless{}{-}} \SpecialCharTok{{-}}\FunctionTok{log}\NormalTok{(}\FunctionTok{runif}\NormalTok{(n)) }\SpecialCharTok{/}\NormalTok{ (lambda1 }\SpecialCharTok{*} \FunctionTok{exp}\NormalTok{(lp1))}
\NormalTok{  Ti2 }\OtherTok{\textless{}{-}} \SpecialCharTok{{-}}\FunctionTok{log}\NormalTok{(}\FunctionTok{runif}\NormalTok{(n)) }\SpecialCharTok{/}\NormalTok{ (lambda2 }\SpecialCharTok{*} \FunctionTok{exp}\NormalTok{(lp2))}
\NormalTok{  Ci  }\OtherTok{\textless{}{-}} \SpecialCharTok{{-}}\FunctionTok{log}\NormalTok{(}\FunctionTok{runif}\NormalTok{(n)) }\SpecialCharTok{/}\NormalTok{ lambda\_c}

\NormalTok{  ti }\OtherTok{\textless{}{-}} \FunctionTok{pmin}\NormalTok{(Ti1, Ti2, Ci)}
\NormalTok{  di }\OtherTok{\textless{}{-}} \FunctionTok{as.integer}\NormalTok{(Ti1 }\SpecialCharTok{\textless{}=}\NormalTok{ Ci }\SpecialCharTok{|}\NormalTok{ Ti2 }\SpecialCharTok{\textless{}=}\NormalTok{ Ci)}
\NormalTok{  di[}\FunctionTok{which}\NormalTok{(Ti2 }\SpecialCharTok{\textless{}=}\NormalTok{ Ti1 }\SpecialCharTok{\&}\NormalTok{ Ti2 }\SpecialCharTok{\textless{}=}\NormalTok{ Ci)] }\OtherTok{\textless{}{-}} \DecValTok{2}
\end{Highlighting}
\end{Shaded}
\end{frame}

%%% Final slide with contact information
\thanksframe{Thank you for your attention}

%%% bib?
\begin{frame}{Bibliography}

\printbibliography[heading=none]

\end{frame}

\end{document}
