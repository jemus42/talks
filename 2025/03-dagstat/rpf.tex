%%%%%%%%%%%%%%%%%%%%%%%%%%%%%%%%%%%%%%%%%%%%%%%%%%%%%
% Primary document settings
%%%%%%%%%%%%%%%%%%%%%%%%%%%%%%%%%%%%%%%%%%%%%%%%%%%%%

\documentclass[aspectratio=169,12pt]{beamer}

\usepackage{ifxetex,ifluatex}
\ifnum 0\ifxetex 1\fi\ifluatex 1\fi=0 % if pdftex
  \usepackage[T1]{fontenc}
  \usepackage[utf8]{inputenc}
  \usepackage{textcomp} % provide euro and other symbols
\else % if luatex or xetex
  \usepackage{unicode-math}
  \defaultfontfeatures{Scale=MatchLowercase}
  \defaultfontfeatures[\rmfamily]{Ligatures=TeX,Scale=1}

\usepackage[
    backend=biber,
    natbib=true,
    style=authoryear-comp,
    %bibstyle=authoryear,
    %autocite=footnote,
    %style=authoryear,
    sorting=nyt,
    %sortlocale=de_DE,
    sortlocale=en_US,
    url=false,
    doi=true,
]{biblatex}

\addbibresource{references.bib}

\usepackage{fancyqr}

%%%%%%%%%%%%%%%%%%%%%%%%%%%%%%%%%%%%%%%%%%%%%%%%%%%%%%%%%%%%%%
% Extra stuff for Rmarkdown to work (code blocks)
%%%%%%%%%%%%%%%%%%%%%%%%%%%%%%%%%%%%%%%%%%%%%%%%%%%%%%%%%%%%%%
\providecommand{\tightlist}{%
  \setlength{\itemsep}{2pt}\setlength{\parskip}{0pt}}

\usepackage{graphicx}
\makeatletter
% New in pandoc https://github.com/jgm/pandoc-templates/commit/6c0e7b0a4f990debcd38b5c3bd8599193ae8f5a6#diff-f218051b4ca8f740a9f585a149101d4a3025037c568b391b5216edf7b14cfadc
\newsavebox\pandoc@box
\newcommand*\pandocbounded[1]{% scales image to fit in text height/width
\sbox\pandoc@box{#1}%
\Gscale@div\@tempa{\textheight}{\dimexpr\ht\pandoc@box+\dp\pandoc@box\relax}%
\Gscale@div\@tempb{\linewidth}{\wd\pandoc@box}%
\ifdim\@tempb\p@<\@tempa\p@\let\@tempa\@tempb\fi% select the smaller of both
\ifdim\@tempa\p@<\p@\scalebox{\@tempa}{\usebox\pandoc@box}%
\else\usebox{\pandoc@box}%
\fi%
}
% Set default figure placement to htbp
\def\fps@figure{htbp}
\makeatother

%%%%%%%%%%%%%%%%%%%%%%%%%%
% kableExtra stuff (tables)
%%%%%%%%%%%%%%%%%%%%%%%%%%
\usepackage{booktabs}
\usepackage{longtable}
\usepackage{array}
\usepackage{multirow}
\usepackage{xcolor}
\usepackage{wrapfig}
\usepackage{float}
\usepackage{colortbl}
\usepackage{pdflscape}
\usepackage{tabu}
\usepackage{threeparttable}
\usepackage{threeparttablex}
\usepackage[normalem]{ulem}
\usepackage{makecell}



%%%%%%%%%%%%%%%%%%%%%%%%%%
% Main document
%%%%%%%%%%%%%%%%%%%%%%%%%%

\usetheme[]{BIPS}
%\usetheme[german]{BIPS}  % for the German version
%\usetheme[fira]{BIPS} % English with the Fira font
%\usetheme[german,fira]{BIPS} % German with the Fira font

% Note that for the Fira font, you need to use
% XeLaTeX instead of pdfLaTeX. You can find this in
% most interfaces

\title{Random Planted Forest\\
A Directly Interpretable Tree Ensemble}
\subtitle{Awwww yeah}
\author{Meyer, J. T.\inst{5} \and Burk, L.\inst{1,2,3,4} \and Hiabu,
M.\inst{6} \and Mammen, E.\inst{5}}
\date{}
\contactauthor{Lukas Burk}
\occasion{DAGStat 2025 --- March 27th, 2025}
\email{burk@leibniz-bips.de}
\institute{\textsuperscript{1}Leibniz Institute for Prevention Research
and Epidemiology -- BIPS \and \textsuperscript{2}LMU Munich
\quad \textsuperscript{3}University of
Bremen \and \textsuperscript{4}Munich Center for Machine Learning
(MCML) \and \textsuperscript{5}Heidelberg University
\quad \textsuperscript{6}University of Copenhagen}


%%% Title Page
\setbeamertemplate{title page}{
	\usebeamercolor{title page}
	\begin{tikzpicture}
		\useasboundingbox (1,0) rectangle (\the\paperwidth,\the\paperheight);
		\node[font=\usebeamerfont{title}, color=BIPSBlue, text width=14cm, align=center] at (\paperwidth*.5,7) {\inserttitle} ;
		% \node[align=center, font=\usebeamerfont{subtitle}, color=BIPSBlue] at (\paperwidth*.5, 5.5) {\insertsubtitle};
		\node[align=center, font=\usebeamerfont{author}, color=BIPSBlue] at (\paperwidth*.5, 5) {\insertauthor};
		\node[align=center, font=\usebeamerfont{institute}, color=BIPSTextGray] at (\paperwidth*.5, 3) {\textsuperscript{1}Leibniz
Institute for Prevention Research and Epidemiology --
BIPS \\ \textsuperscript{2}LMU Munich
\quad \textsuperscript{3}University of
Bremen \\ \textsuperscript{4}Munich Center for Machine Learning
(MCML) \\ \textsuperscript{5}Heidelberg University
\quad \textsuperscript{6}University of Copenhagen};
		\node[align=center, font=\usebeamerfont{date}, color=BIPSTextGray] at (\paperwidth*.5, 1.5) {\insertdate};
		\node[align=center, font=\usebeamerfont{date}, color=BIPSTextGray] at (\paperwidth*.5, 1) {DAGStat
2025 --- March 27th, 2025};
	\end{tikzpicture}
}


\usepackage{tikz}
\usepackage{amsmath,amssymb}

\definecolor{brewer1}{rgb}{0.105882,0.619608,0.466667}
\definecolor{brewer2}{rgb}{0.85098,0.372549,0.00784314}
\definecolor{brewer3}{rgb}{0.458824,0.439216,0.701961}
\definecolor{brewer4}{rgb}{0.905882,0.160784,0.541176}

\usetikzlibrary{arrows.meta, shapes, positioning, calc}

\tikzset{
  root/.style={rectangle, rounded corners, minimum width=2cm, minimum height=0.8cm, text centered, draw=black, fill=brewer2, line width=1pt},
  internalnode/.style={rectangle, minimum width=1.7cm, minimum height=0.7cm, text centered, draw=black, fill=gray!2},
  decision/.style={ellipse, minimum width=1.7cm, minimum height=0.7cm, text centered, draw=black, fill=gray!2},
  rpfunique/.style={rectangle, minimum width=1.7cm, minimum height=0.7cm, text centered, draw=black, fill=brewer1},
  rpfuniquedecision/.style={ellipse, minimum width=1.7cm, minimum height=0.7cm, text centered, draw=black, fill=brewer1},
  interaction/.style={rectangle, minimum width=1.7cm, minimum height=0.7cm, text centered, draw=black, fill=brewer3},
  arrow/.style={thick,->,>=stealth},
  label/.style={font=\scriptsize}
}


\begin{document}
\addtocounter{framenumber}{-1}
\frame{\maketitle}

% \setcounter{framenumber}{1}


\begin{frame}{Motivation}
\phantomsection\label{motivation}
\begin{itemize}[<+->]
\tightlist
\item
  Individual decision trees: Easy to interpret
\item
  Random Forest (RF): Less so
\item
  \textbf{Desirable properties}:

  \begin{itemize}[<+->]
  \tightlist
  \item
    Meaningful feature importance rather than rankings
  \item
    Quantification of main- \emph{and interaction} effects
  \end{itemize}
\item
  Additive models (LM, GAM, \ldots) can provide both
\item
  → Need to manually specify interactions in model fit
\end{itemize}

\textbf{Random Planted Forest} (RPF): Additive Random Forest
\end{frame}

\begin{frame}{Functional ANOVA Expansion}
\phantomsection\label{functional-anova-expansion}
\begin{itemize}[<+->]
\tightlist
\item
  Setting: Regression with target \(Y_i \in \mathbb{R}^p\) and feature
  vector \(\mathbf{x_i}\)
\item
  Decompose prediction
  \(\mathbb{E}(Y_i | X = \mathbf{x}_i) = \hat{m}(\mathbf{x_i})\) into

  \begin{itemize}[<+->]
  \tightlist
  \item
    Average prediction \(\hat{m}_{0}\) (``intercept'')
  \item
    Terms \(\hat{m}_S\) with \(S \subseteq \{1, \ldots, s\}\)
  \end{itemize}
\end{itemize}

\pause

\begin{align*}
\hat{m}(\mathbf{x}) & = \hat{m}_{0} \\
& + \underbrace{\hat{m}_1(x_1) + \hat{m}_2(x_2) + \hat{m}_3(x_3)}_{\text{Main effects}} \\
& + \underbrace{\hat{m}_{1,2}(x_1,x_2) + \hat{m}_{1,3}(x_1,x_3) + \hat{m}_{2,3}(x_2,x_3)}_{\text{2nd order interactions}} \\ 
& + \underbrace{\hat{m}_{1,2,3}(x_1,x_2,x_3)}_{\text{3rd order interaction}}
\end{align*}
\end{frame}

\begin{frame}{Trees in Random Forest}
\phantomsection\label{trees-in-random-forest}
\begin{figure}[H]

{\centering \includegraphics[width=\linewidth,height=0.75\textheight,keepaspectratio]{tree-cart.pdf}

}

\caption{CARTlike}

\end{figure}%
\end{frame}

\begin{frame}{Planted Trees}
\phantomsection\label{planted-trees}
\begin{figure}

\centering{

\includegraphics[width=\linewidth,height=0.75\textheight,keepaspectratio]{tree-planted-simple.pdf}

}

\caption{\label{fig-planted}Planted}

\end{figure}%
\end{frame}

\begin{frame}{Planted Trees}
\phantomsection\label{planted-trees-1}
\begin{figure}

\centering{

\includegraphics[width=\linewidth,height=0.75\textheight,keepaspectratio]{tree-planted-large.pdf}

}

\caption{\label{fig-planted}Planted-large}

\end{figure}%
\end{frame}

\begin{frame}{Key Differences of RPF to RF}
\phantomsection\label{key-differences-of-rpf-to-rf}
\begin{itemize}[<+->]
\tightlist
\item
  Splits some nodes multiple times (→ non-binary trees)
\item
  Keeps track of features involved in split
\item
  Degree of interaction can be constrained
\item
  Stopping after adjustable number of splits
\item
  Prediction is average of additive \(\hat{m}_S\) estimates
\end{itemize}
\end{frame}

\begin{frame}[fragile]{Application Example}
\phantomsection\label{application-example}
\begin{itemize}[<+->]
\tightlist
\item
  \texttt{Bikeshare} regression dataset \footnote<.->{from
    \href{https://archive.ics.uci.edu/dataset/275/bike+sharing+dataset}{UCI
    ML repository}}
\item
  Target \texttt{bikers}: Number of bikers on a given day in 2011/2012
\item
  Focus on 3 features for example

  \begin{itemize}[<+->]
  \tightlist
  \item
    \texttt{hour} of day \(\in \{0, 1, \ldots, 23\}\)
  \item
    \texttt{temp} normalized temperature \(\in [0, 1]\)
  \item
    \texttt{workingday} binary → \{\texttt{workingday},
    \texttt{no\ workingday}\}
  \end{itemize}
\item
  Average prediction: \(\hat{m}_0 \approx\) 143.7
\end{itemize}
\end{frame}

\begin{frame}{Main Effects}
\phantomsection\label{main-effects}
\begin{columns}[T]
\begin{column}{0.33\linewidth}
\begin{center}
\includegraphics[width=\linewidth,height=0.75\textheight,keepaspectratio]{rpf_files/figure-beamer/main-hr-1.png}
\end{center}
\end{column}

\pause

\begin{column}{0.33\linewidth}
\begin{center}
\includegraphics[width=\linewidth,height=0.75\textheight,keepaspectratio]{rpf_files/figure-beamer/main-temp-1.png}
\end{center}
\end{column}

\pause

\begin{column}{0.33\linewidth}
\begin{center}
\includegraphics[width=\linewidth,height=0.75\textheight,keepaspectratio]{rpf_files/figure-beamer/main-workingday-1.png}
\end{center}
\end{column}
\end{columns}
\end{frame}

\begin{frame}{Hour \(\times\) Working Day: ``Rush Hour Effect''}
\phantomsection\label{hour-times-working-day-rush-hour-effect}
\begin{center}
\includegraphics[width=\linewidth,height=0.75\textheight,keepaspectratio]{rpf_files/figure-beamer/twoway-hr-workingday-1.png}
\end{center}
\end{frame}

\begin{frame}{More 2nd Order Interactions}
\phantomsection\label{more-2nd-order-interactions}
\begin{columns}[T]
\begin{column}{0.5\linewidth}
\begin{center}
\includegraphics[width=\linewidth,height=0.75\textheight,keepaspectratio]{rpf_files/figure-beamer/unnamed-chunk-1-1.png}
\end{center}
\end{column}

\pause

\begin{column}{0.5\linewidth}
\begin{center}
\includegraphics[width=\linewidth,height=0.75\textheight,keepaspectratio]{rpf_files/figure-beamer/unnamed-chunk-2-1.png}
\end{center}
\end{column}
\end{columns}
\end{frame}

\begin{frame}{3rd Order Interaction}
\phantomsection\label{rd-order-interaction}
\begin{center}
\includegraphics[width=\linewidth,height=0.75\textheight,keepaspectratio]{rpf_files/figure-beamer/unnamed-chunk-3-1.png}
\end{center}
\end{frame}

\begin{frame}{Feature Importance}
\phantomsection\label{feature-importance}
\begin{itemize}[<+->]
\tightlist
\item
  Average of absolute values of term of interest
\end{itemize}

\pause

\[\mathrm{VI}_S(X) = \frac{1}{n} \sum_{i=1}^n |\hat{m}_S(\mathbf{x}_i)|\]

\begin{itemize}[<+->]
\tightlist
\item
  Unlike RF Feature importance:

  \begin{itemize}[<+->]
  \tightlist
  \item
    Scores \emph{per interaction} term
  \item
    Importance scores on same scale as prediction
  \end{itemize}
\end{itemize}
\end{frame}

\begin{frame}{Feature Importance per Main Term}
\phantomsection\label{feature-importance-per-main-term}
\begin{center}
\includegraphics[width=\linewidth,height=0.75\textheight,keepaspectratio]{rpf_files/figure-beamer/vi-plot-1.png}
\end{center}
\end{frame}

\begin{frame}{Feature Importance per All Terms}
\phantomsection\label{feature-importance-per-all-terms}
\begin{center}
\includegraphics[width=\linewidth,height=0.75\textheight,keepaspectratio]{rpf_files/figure-beamer/vi-plot-thresh-1.png}
\end{center}
\end{frame}

\begin{frame}{Feature Importance by Order of Interaction}
\phantomsection\label{feature-importance-by-order-of-interaction}
\begin{center}
\includegraphics[width=\linewidth,height=0.75\textheight,keepaspectratio]{rpf_files/figure-beamer/vi-plot-degree-1.png}
\end{center}
\end{frame}

\begin{frame}[fragile]{Related work}
\phantomsection\label{related-work}
Glex\footnote<.->{\textcite{hiabu2023glex}}: Same ANOVA decomposition
but generally for tree-based methods (e.g.~XGBoost)

\begin{itemize}[<+->]
\tightlist
\item
  \textbf{Idea}:

  \begin{itemize}[<+->]
  \tightlist
  \item
    Fit XGBoost with more shallow trees (e.g.~\texttt{max\_depth\ =\ 4}
    \(\leftrightarrow\) \texttt{max\_interaction\ =\ 4})
  \item
    Extract component-wise predictions from tree structure
  \end{itemize}
\item
  \textbf{Benefit}: Use existing / well known / well optimzied algorithm
\item
  \textbf{Drawback}: Computationally intensive post-hoc computation
\end{itemize}
\end{frame}

\begin{frame}{No Free Lunch}
\phantomsection\label{no-free-lunch}
Better interpretibility → worse predictive performance?

\pause

\begin{itemize}[<+->]
\tightlist
\item
  Benchmark comparing RPF with XGBoost, RF incl.~tuning
\item
  28 datasets from OpenML-CTR23 regression benchmark suite
  \footnote<.->{\textcite{fischer2023openmlctr}}
\item
  Also comparing XGBoost / RPF with interactions restrained to 2
\item
  Generally XGBoost best, RPF and RF closely behind
\end{itemize}
\end{frame}

\begin{frame}{Benchmark Results (Aggregated)}
\phantomsection\label{benchmark-results-aggregated}
\begin{itemize}[<+->]
\tightlist
\item
  Root-Relative Squared Error (RRSE)
  \(\sqrt{\tfrac{SSE(Y, \hat{Y})}{SSE(Y, \bar{Y})}}\)
\item
  Featureless model scores 1, perfect score 0
\end{itemize}

\begin{center}
\includegraphics[width=\linewidth,height=0.75\textheight,keepaspectratio]{rpf_files/figure-beamer/bm-aggr-rrse-1.png}
\end{center}
\end{frame}

\begin{frame}{Results for Selected Tasks}
\phantomsection\label{results-for-selected-tasks}
\begin{center}
\includegraphics[width=\linewidth,height=0.75\textheight,keepaspectratio]{rpf_files/figure-beamer/bm-scores-sel-1.png}
\end{center}
\end{frame}

\begin{frame}{Summary}
\phantomsection\label{summary}
\textbf{Random Planted Forests} = Additive Random Forests

\begin{itemize}[<+->]
\tightlist
\item
  (+) Interpretability on global and local perspective
\item
  (+) Interpretable on same scale as target
\item
  (\textasciitilde) Predictive performance worse but similar to
  comparable algorithms
\item
  (-) Computationally heavy for large data
\item
  (+) R package available \footnote<.->{\href{https://github.com/PlantedML/randomPlantedForest}{github.com/PlantedML/randomPlantedForest}}
\end{itemize}
\end{frame}

%%% Final slide with contact information
\thanksframe{Thank you for your attention!}

%%% bib?
\begin{frame}[allowframebreaks]{References}
  \printbibliography[heading=none]
\end{frame}

\end{document}
